%%%%%%%%%%%%%%%%%%%%%%%%%%%%%%%%%%%%%%%%%
% Medium Length Professional CV
% LaTeX Template
% Version 2.0 (8/5/13)
%
% This template has been downloaded from:
% http://www.LaTeXTemplates.com
%
% Original author:
% Trey Hunner (http://www.treyhunner.com/)
% Modified by N. Wisnoski
%
% Important note:
% This template requires the resume.cls file to be in the same directory as the
% .tex file. The resume.cls file provides the resume style used for structuring the
% document.
%
%%%%%%%%%%%%%%%%%%%%%%%%%%%%%%%%%%%%%%%%%

%-------------------------------------------------------------------------------
%	PACKAGES AND OTHER DOCUMENT CONFIGURATIONS
%-------------------------------------------------------------------------------

\documentclass{resume} % Use the custom resume.cls style

\usepackage[left=.75in,top=1in,right=.75in,bottom=1in]{geometry} % Document margins
\usepackage{fancyhdr}
\usepackage{datetime}
\usepackage{eurosym}
\usepackage{hyperref}
\usepackage{hanging}
\usepackage{csquotes}
\usepackage{tabulary}
\usepackage[leftmargin=2.5em, listparindent=-1em,  labelsep = 1em]{etaremune} % to make hanging indent, add itemindent=-1em,
\newdateformat{mydate}{\THEMONTH/\THEYEAR}
\fancyhf{} % sets both header and footer to nothing
\renewcommand{\headrulewidth}{0pt}
\rfoot{\mydate\today}
\cfoot{\thepage}
\newcommand{\Star}{\ensuremath{^*}\kern-\scriptspace}

\name{\sc Nathan I. Wisnoski} % Your name
\address{Mississippi State University \\ 295 Lee Blvd, Mississippi State, MS 39762} % Your address
\address{nathan.wisnoski@msstate.edu \\ www.nathanwisnoski.com} % Your phone number and email

\begin{document}

%-------------------------------------------------------------------------------
%	EDUCATION SECTION
%-------------------------------------------------------------------------------

\begin{rSection}{Education}

{\bf Indiana University, Bloomington} \hfill {2014 -- 2020} \\
Ph.D in Biology -- Evolution, Ecology, and Behavior \\
Minor in Environmental Studies, O'Neill School of Public and Environmental Affairs

{\bf The University of Texas at Austin} \hfill {2009 -- 2013} \\
B.S. in Biology -- Ecology, Evolution, and Behavior \\
Minor in Business, McCombs School of Business

\end{rSection}
\bigskip

%-------------------------------------------------------------------------------
%	RESEARCH EXPERIENCE SECTION
%-------------------------------------------------------------------------------

\begin{rSection}{Experience}

\begin{rSubsection}{Assistant Professor}{August 2022 -- Present}{Mississippi State University, Department of Biological Sciences}{Starkville, MS}
%\item
\end{rSubsection}

\begin{rSubsection}{Postdoctoral Researcher}{July 2020 -- August 2022}{University of Wyoming, Wyoming Geographic Information Science Center}{Laramie, WY}
%\item Albeke Lab: community assembly, biogeography, individual-based modeling 
%\item Shoemaker Lab: coexistence theory, metacommunity theory
\end{rSubsection}

\begin{rSubsection}{Graduate Research \& Teaching Assistant}{August 2014 -- May 2020}{Indiana University, Department of Biology}{Bloomington, IN}
%\item Lennon Lab: metacommunity ecology, aquatic microbial ecology, synthesis research
\end{rSubsection}

\begin{rSubsection}{Lab \& Field Technician}{Spring 2014}{University of Texas, Department of Integrative Biology}{Austin, TX}
%\item Hawkes Lab: soil microbial ecology, ecosystem ecology
\end{rSubsection}

\end{rSection}
\bigskip
%-------------------------------------------------------------------------------
%   Manuscripts
%-------------------------------------------------------------------------------
\begin{rSection}{Publications}

% {\it Peer-reviewed: }

\begin{etaremune}
\item {\bf Wisnoski, N.I.}, R. Andrade, M.C.N. Castorani, C.P. Catano, A. Compagnoni, T. Lamy, N.K. Lany, L. Marazzi, S. Record, A.C. Smith, C.M. Swan, J.D. Tonkin, N.M. Voelker, P.L. Zarnetske, and E.R. Sokol. In press. Diversity-stability relationships become decoupled across spatial scales: a synthesis of organism and ecosystem types. Ecology

\item {\bf Wisnoski, N.I.} and J.T. Lennon. 2023. Scaling up and down: movement ecology for microorganisms. Trends in Microbiology 31(3):242-253.

\item Mobilian, C., {\bf N.I. Wisnoski}, J.T. Lennon, M. Alber, S. Widney, C.B. Craft. 2023. Differential effects of press vs. pulse seawater intrusion on microbial communities of a tidal freshwater marsh. Limnology and Oceanography Letters 8(1):151-161.

\item {\bf Wisnoski, N.I.} and L.G. Shoemaker. 2022. Seed banks alter metacommunity diversity: the interactive effects of competition, dispersal, and dormancy. Ecology Letters 25(4):740-753.

\item Lamy, T., {\bf N.I. Wisnoski}, R. Andrade, M.C.N. Castorani, A. Compagnoni, N. Lany, L. Marazzi, S. Record, C.M. Swan, J.D. Tonkin, N. Voelker, S. Wang, P.L. Zarnetske, and E.R. Sokol. 2021. The dual nature of metacommunity variability. Oikos 130(12):2078-2092.

\item {\bf Wisnoski, N.I.} and J.T. Lennon. 2021. Stabilizing role of seed banks and the maintenance of bacterial diversity. Ecology Letters 24(11):2328-2338.

\item Shoemaker, L.G., J.A. Walter, L.A. Gherardi, M.H. DeSiervo, and {\bf N.I. Wisnoski}. 2021 Writing mathematical ecology: a guide for authors and readers. Ecosphere 12(8):e03701.

\item Graham, E.B., C. Averill, B. Bond-Lamberty, J.E. Knelman, S. Krause, A.L. Peralta, A. Shade, A.P. Smith, S. Cheng, N. Fanin, C. Freund, P.E. Garcia, S.M. Gibbons, M.W. Van Goethem, M.B. Guebila, J. Kemppinen, R. Nowicki, J.G. Pausas, S. Reed, J. Rocca, A. Sengupta, D. Sihi, M. Simonin, M. Słowiński, S. Spawn, I. Sutherland, J. Tonkin, {\bf N.I. Wisnoski}, S.C. Zipper, and Contributor Consortium. 2021. Toward a unifying framework of disturbance ecology through crowdsourced science. Frontiers in Ecology and Evolution 9:76.

\item Record, S., N.M. Voelker, P.L. Zarnetske, {\bf N.I. Wisnoski}, J.D. Tonkin, C.M. Swan, L. Marazzi, N. Lany, T. Lamy, A. Compagnoni, M.C.N. Castorani, R. Andrade, and E.R. Sokol. 2021. Novel insights to be gained from applying metacommunity theory to long-term, spatially replicated biodiversity data. Frontiers in Ecology and Evolution 8:479.

\item {\bf Wisnoski, N.I.} and J.T. Lennon. 2021. Microbial community assembly in a multi-layer dendritic metacommunity. Oecologia 195:13-24.

\item {\bf Wisnoski, N.I.}, M.E. Muscarella, M.L. Larsen, A.L. Peralta, and J.T. Lennon. 2020. Metabolic insight into bacterial community assembly across ecosystem boundaries. Ecology 101(4):e02968.

\item Mueller, E.A., {\bf N.I. Wisnoski}, A.L. Peralta, and J.T. Lennon. 2020. Microbial rescue effects: how microbiomes can save hosts from extinction. Functional Ecology 30(10):2055-2064.

\item Ward, A.S., S.M. Wondzell, N.M. Schmadel, S. Herzog, J.P. Zarnetske, V. Baranov, P.J. Blaen, N. Brekenfeld, R. Chu, R. Derelle, J. Drummond, J.H. Fleckenstein, V. Garayburu-Caruso, E. Graham, D. Hannah, C.J. Harman, J. Hixson, J.L.A. Knapp, S. Krause, M.J. Kurz, J. Lewandowski, A. Li, E. Marti, M. Miller, A.M. Milner, K. Neil, L. Orsini, A.I. Packman, S. Plont, L. Renteria, K. Roche, T. Royer, C. Segura, J. Stegen, J. Toyoda, J. Wells, and {\bf N.I. Wisnoski}. 2019. Spatial and temporal variation in river corridor exchange across a 5th-order mountain stream network. Hydrology and Earth System Sciences 23:5199-5225.

\item Ward, A.S., M.J. Kurz, N.M. Schmadel, J.L.A. Knapp, P.J. Blaen, C.J. Harman, J.D. Drummond, D.M. Hannah, S. Krause, A. Li, E. Marti, A. Milner, M. Miller, K. Neil, S. Plont, A.I. Packman, {\bf N.I. Wisnoski}, S.M. Wondzell, and J.P. Zarnetske. 2019. Solute transport and transformation in an intermittent, headwater mountain stream with diurnal discharge fluctuations. Water 11(11):2208.

\item Ward, A.S., J.P. Zarnetske, V. Baranov, P.J. Blaen, N. Brekenfeld, R. Chu, R. Derelle, J. Drummond, J.H. Fleckenstein, V. Garayburu-Caruso, E. Graham, D. Hannah, C.J. Harman, S. Herzog, J. Hixson, J.L.A. Knapp, S. Krause, M.J. Kurz, J. Lewandowski, A. Li, E. Marti, M. Miller, A.M. Milner, K. Neil, L. Orsini, A.I. Packman, S. Plont, L. Renteria, K. Roche, T. Royer, N.M. Schmadel, C. Segura, J. Stegen, J. Toyoda, J. Wells, {\bf N.I. Wisnoski}, and S.M. Wondzell. 2019. Co-located contemporaneous mapping of morphological, hydrological, chemical, and biological conditions in a 5th-order mountain stream network, Oregon, USA. Earth System Science Data 11:1567-1581.

\item {\bf Wisnoski, N.I.}, M.A. Leibold, and J.T. Lennon. 2019. Dormancy in metacommunities. The American Naturalist 194(2):135-151.

\end{etaremune}

\bigskip

{\it Book Reviews:}
\begin{etaremune}
\item {\bf Wisnoski, N.I.} and J.T. Lennon. 2016. \enquote{Principles of Microbial Diversity} by James W. Brown. The Quarterly Review of Biology 91(1): 98-99.

\end{etaremune}
\end{rSection}

\bigskip

%-------------------------------------------------------------------------------
%   FUNDING/AWARDS/Fellowships
%-------------------------------------------------------------------------------
\newpage
\begin{rSection}{Grants}
{\def\arraystretch{1.5}
\begin{tabulary}{0.95\textwidth}{lcL}
2023--2025 & \$614,757 & Mississippi Based RESTORE Act Center of Excellence (MBRACE). \textit{Transport and fate of bacterial communities in Mississippi coastal ecosystems}. PI: {\bf N.I. Wisnoski}. \\

2021--2022 & \$65,888 & Wyoming EPSCoR: Microbial Ecology Collaborative. \textit{Microbial community assembly in dendritic networks: disentangling the roles of local environmental heterogeneity, stochasticity, and dispersal}. PIs: {\bf N.I. Wisnoski} and S.E. Albeke.\\

2016--2018 & \$76,000 & NSF LTER Network Communications Office (NCEAS). \textit{A synthesis to identify how metacommunity dynamics mediate community responses to disturbance across the ecosystems represented in the LTER network}. PI: E.R. Sokol, Co-PIs: C.M. Swan, {\bf N.I. Wisnoski}.\\

2015 & \$5,400 & Indiana University. Sustainability Research Development Grant.

\end{tabulary}
}
\end{rSection}

\bigskip

\begin{rSection}{Fellowships and Awards}

\begin{Award}{Louise Constable Hoover Fellowship, IU Biology}{\$2000}{2019}
\end{Award}

\begin{Award}{Travel Award, Association for the Sciences of Limnology and Oceanography}{\$606}{2018}
\end{Award}

\begin{Award}{Travel Award, ESA Microbial Ecology Section}{\$600}{2017}
\end{Award}

\begin{Award}{George W. Brackenridge Fellowship, IU Biology}{\$2000}{2016}
\end{Award}

\begin{Award}{Travel Award, International Society for Microbial Ecology}{\euro{}300}{2016}
\end{Award}

\begin{Award}{Travel Award, Honorable Mention, ESA Microbial Ecology Section}{\$150}{2016}
\end{Award}

\begin{Award}{Departmental Research Recruitment Fellowship, IU Biology}{ }{2014}
\end{Award}

\end{rSection}

\bigskip
%-------------------------------------------------------------------------------
%   POSTERS / PRESENTATIONS
%-------------------------------------------------------------------------------
\begin{rSection}{Invited Seminars}
{\def\arraystretch{1}
\begin{tabulary}{\textwidth}{lL}
2022 & Swiss Federal Institute of Aquatic Science and Technology (Eawag), Department of Aquatic Ecology\\

2022 & Mississippi State University, Department of Biochemistry, Molecular Biology, Entomology \& Plant Pathology\\

2022 & University of Bonn, Hausdorff Research Insitute for Mathematics\\

2021 & Mississippi State University, Department of Biological Sciences\\

2020 & University of Wyoming, Department of Botany\\
\end{tabulary}
}
\end{rSection}

\bigskip

\begin{rhangSection}{Conference Talks and Posters}
\emph{\small Note: * = Presenter}

\begin{Presentation}{{\bf Wisnoski, N.I.}\Star\ and J.T. Lennon}{2023}{Microbial movement ecology: linking motility and dispersal across scales.}{American Society for Microbiology: Microbe}{Houston, TX [Canceled due to illness]}
\end{Presentation}

\begin{Presentation}{{\bf Wisnoski, N.I.}\Star\ and L.G. Shoemaker}{2022}{Seed banks in metacommunities: when and how do they maintain biodiversity?}{Ecological Society of America Annual Meeting}{Montreal, QC, Canada [Canceled due to illness]}
\end{Presentation}
    
\begin{Presentation}{Davidson, J.L.\Star, K. McKnight, M.H. DeSiervo, C.M. Werner, M.C. Szojka, {\bf N.I. Wisnoski}, E.W. Seabloom, L.G. Shoemaker}{2022}{Community synchrony, stability, and biodiversity responses to global change: evaluating short-term versus multi-decadal responses.}{Ecological Society of America Annual Meeting}{Montreal, QC, Canada}
\end{Presentation}

\begin{Presentation}{von Eggers, J.\Star, D. Groff, {\bf N. Wisnoski}, J. Calder, A. Krist, B. Shuman}{2022}{Controls on biogeography and assembly of lake sedimentary microbial communities.}{Joint Aquatic Sciences Meeting}{Grand Rapids, MI}
\end{Presentation}

\begin{Presentation}{{\bf Wisnoski, N.I.}\Star, M.E. Muscarella, M.L. Larsen, A.L. Peralta, and J.T. Lennon}{2021}{Metabolic insight into bacterial community assembly across ecosystem boundaries.}{Ecological Society of America Annual Meeting}{Long Beach, CA (Virtual)}
    \end{Presentation}
    
\begin{Presentation}{{\bf Wisnoski, N.I.} and E.R. Sokol\Star}{2021}{Diversity-stability relationships across spatial scales: aggregate and compositional variability correlate with different dimensions of metacommunity diversity.}{Ecological Society of America Annual Meeting}{Long Beach, CA (Virtual)}
\end{Presentation}

\begin{Presentation}{{\bf Wisnoski, N.I.}\Star, E.R. Sokol, R. Andrade, M.C.N. Castorani, C.P. Catano, A. Compagnoni, T. Lamy, N.K. Lany, L. Marazzi, S. Record, A.C. Smith, C.M. Swan, J.D. Tonkin, N.M. Voelker, P.L. Zarnetske}{2019}{Patterns and drivers of stability in long-term metacommunity data.}{Ecological Society of America Annual Meeting}{Louisville, KY}
\end{Presentation}
  
\begin{Presentation}{Sokol, E.R.\Star, {\bf N.I. Wisnoski}, and C.M. Swan}{2019}{Insights from the synthesis of long-term biodiversity data: resources and tools available to community ecologists.}{Ecological Society of America Annual Meeting}{Louisville, KY}
\end{Presentation}
  
\begin{Presentation}{{\bf Wisnoski, N.I.}\Star, M.A. Leibold, J.T. Lennon}{2019}{Dormancy in metacommunities: when can temporal dispersal maintain diversity in variable landscapes?}{Society for Freshwater Science Annual Meeting}{Salt Lake City, UT}
\end{Presentation}
  
\begin{Presentation}{Ward, A. S.\Star, C.J. Harman, N.M. Schmadel, M.J. Kurz, P. Blaen, S.M. Wondzell, J.D. Drummond, D.M. Hannah, J.L. Knapp, S. Krause, A. Li, E.R. Martí, M. Miller, A. Milner, K. Neil, S. Plont, K.R. Roche, A.I. Packman, {\bf N. Wisnoski}, J.P. Zarnetske}{2018}{How do evapotranspiration-driven discharge fluctuations alter reach-scale ecosystem function?}{American Geophysical Union, Fall Meeting}{Washington, D.C.}
\end{Presentation}
  
\begin{Presentation}{Ward, A. S.\Star, S. Herzog, S.M. Wondzell, N.M. Schmadel, P. Blaen, J.D. Drummond, D.M. Hannah, C.J. Harman, J.L. Knapp, S. Krause, M.J. Kurz, A. Li, E. Martí, M. Miller, A. Milner, K. Neil, S. Plont, K.R. Roche, A.I. Packman, {\bf N. Wisnoski}, and J.P. Zarnetske}{2018}{Spatial and temporal relationships between hydrologic forcing, geologic setting, and river corridor exchange in a mountain stream network.}{American Geophysical Union, Fall Meeting}{Washington, D.C.}
\end{Presentation}
  
\begin{Presentation}{Ward, A.S.\Star, S. Herzog, S.M. Wondzell, N. Schmadel, P. Blaen, J. Drummond, D.M. Hannah, C.J. Harman, J. Knapp, S. Krause, M.J. Kurz, A. Li, E. Marti, M. Miller, A. Milner, K. Neil, S. Plont, K. Roche, A.I. Packman, {\bf N. Wisnoski}, and J. Zarnetske}{2018}{How do hydrologic forcing and geologic setting control river corridor exchange in a 5th order mountain stream network?}{Geological Society of America Annual Meeting}{Indianapolis, IN}
\end{Presentation}
  
\begin{Presentation}{{\bf Wisnoski, N.I.}\Star\ and J.T. Lennon}{2018}{Contribution of \enquote{seed banks} to bacterioplankton community dynamics.}{Society for Freshwater Science Annual Meeting}{Detroit, MI}
\end{Presentation}
  
\begin{Presentation}{Sokol, E.R.\Star, {\bf N.I. Wisnoski}, and C.M. Swan}{2018}{Using long-term data to understand when metacommunities respond to disturbance.}{Ecological Society of America Annual Meeting}{New Orleans, LA}
\end{Presentation}

\begin{Presentation}{{\bf Wisnoski, N.I.}\Star, M.E. Muscarella, and J.T. Lennon}{2018}{Dispersal and dormancy across ecosystem boundaries.}{Association for the Sciences of Limnology and Oceanography}{Victoria, BC, Canada}
\end{Presentation}

\begin{Presentation}{{\bf Wisnoski, N.I.}\Star\ and J.T. Lennon}{2017}{Microbial community assembly in dendritic metacommunities.}{Ecological Society of America Annual Meeting}{Portland, OR}
\end{Presentation}
  
\begin{Presentation}{Sokol, E.R.\Star, {\bf N.I. Wisnoski}, C.M. Swan, R. Andrade, H.L. Bateman, A.G. Hope, J. Kominoski, N.K. Lany, L. Marazzi, S.J. Presley, A. Rassweiler, S. Record, M.R. Willig, and P.L. Zarnetske}{2017}{The role of long-term ecological research programs for testing metacommunity theory and understanding biodiversity patterns.}{Ecological Society of America Annual Meeting}{Portland, OR}
\end{Presentation}
  
\begin{Presentation}{Voelker, N.M.\Star, E.R. Sokol, {\bf N.I. Wisnoski}, C.M. Swan, T. Lamy, M.C.N. Castorani, L. Marazzi, A. Compagnoni, J.R. Blanchard, R. Andrade, and N.K. Lany}{2017}{Evaluating the link between metacommunity stability and environmental variability across trophic groups represented at LTER sites.}{Ecological Society of America Annual Meeting}{Portland, OR}
\end{Presentation}

\begin{Presentation}{{\bf Wisnoski, N.I.}\Star\ and J.T. Lennon}{2016}{Community assembly processes differ between surface water and sediment-associated communities in stream networks.}{Ecological Society of America Annual Meeting}{Fort Lauderdale, FL}
\end{Presentation}

\begin{Presentation}{{\bf Wisnoski, N.I.}\Star\ and J.T. Lennon}{2016}{Local and regional processes in stream microbial community assembly (poster).}{International Symposium on Microbial Ecology (ISME 16)}{Montreal, QC, Canada}
\end{Presentation}

\begin{Presentation}{{\bf Wisnoski, N.I.}\Star, A.S. Ward, and J.T. Lennon}{2015}{Bacterial metacommunity structure across a stream network (poster).}{LTER All Scientists Meeting}{Estes Park, CO}
\end{Presentation}

\end{rhangSection}

\bigskip

%-------------------------------------------------------------------------------
%	Teaching SECTION
%-------------------------------------------------------------------------------
\begin{rSection}{Teaching}

\begin{Course}
  {Instructor}{BIO 4993/6993: Community Ecology}{Mississippi State University}{Fall 2023}
\end{Course}

\begin{Course}
  {Instructor}{BIO 3104: Ecology}{Mississippi State University}{Spring 2023}
\end{Course}

\begin{Course}
  {Co-Instructor}{BIOL-Z 620: Quantitative Biodiversity}{Indiana University}{Spring 2017}
\end{Course}

\begin{Course}
  {Associate Instructor}{BIOL-L 111: Foundations of Biology: Diversity, Evolution, and Ecology}{Indiana University}{Spring 2016, Fall 2016, Spring 2018, Spring 2019, Spring 2020}
\end{Course}

\begin{Course}
  {Associate Instructor}{BIOL-L 113: Biology Laboratory}{Indiana University}{Fall 2014, Fall 2017, Fall 2018, Fall 2019}
\end{Course}

\begin{Course}
  {Grader}{BIO 364: Microbial Ecology}{University of Texas}{Spring 2014}
\end{Course}

\begin{Course}
  {Teaching Assistant}{SSC 328M: Biostatistics}{University of Texas}{Spring 2013, Fall 2013}
\end{Course}

\end{rSection}

\bigskip


%-------------------------------------------------------------------------------
%	Workshop Organizer SECTION
%-------------------------------------------------------------------------------
\begin{rSection}{Workshops}

{\bf Invited Participant}. 2022. NSF Workshop on Merging Statistical Theory and Analyses at the Interface of Microbial and ‘Macrobial’ Ecology. Organizers: M. Leibold, P. Peres-Neto, E. Thebault. Montreal, QC, Canada.

{\bf Invited Participant}. 2022. Impacts of dormancy and latency on host-parasite dynamics. Hausdorff Research Institute for Mathematics. Organizer: C. Pokalyuk. Bonn, Germany.

{\bf Lead Organizer}. 2018. Synthesizing long-term community data: questions, challenges, and advances. LTER All Scientists Meeting. Pacific Grove, CA.

\end{rSection}
\bigskip
\newpage
%-------------------------------------------------------------------------------
%	Mentorship Service SECTION
%-------------------------------------------------------------------------------
\begin{rSection}{Mentorship and Service}
{\bf Grant Reviewer, Panelist:}
\begin{itemize}
  \item {\em Long-Term Ecological Research (LTER) Network}: Synthesis Working Groups.
\end{itemize}

{\bf Grant Reviewer, Ad Hoc:}
\begin{itemize}
  \item {\em National Science Foundation (NSF)}: Division of Environmental Biology.
\end{itemize}

{\bf Peer Reviewer:}
\begin{itemize}
  \item {\em Journals}: American Naturalist, Applied and Environmental Microbiology, Aquatic Ecology, Austral Ecology, Biogeosciences, Biology Letters, BioScience, Ecography, Ecological Indicators, Ecology, Ecology Letters, Ecosystem Health and Sustainability, Environmental Microbiology, FEMS Microbial Ecology, Freshwater Science, Global Ecology and Biogeography, The ISME Journal, Journal of Biogeography, Journal of Ecology, Microbial Ecology, mSystems, and New Phytologist.
  \item {\em Books}: Kirchman (2018) Processes in Microbial Ecology (2nd Ed.)
\end{itemize}

{\bf STEM Mentorship:}
\begin{itemize}
  \item Undergraduate Mentees: Luke Pryke, Mollie Carrison, Tajinder Singh
  \item Summer REU Mentees: Jaylen Beatty, Mary Wallace, SydneyEllen Gooding
  \item High School Mentees: Dakayla Calhoun, Samuel Iwu, Ian Schowe
\end{itemize}

{\bf Outreach:}
\begin{itemize}
  \item {Coordinator, High School Riverwatch Sampling} \hfill Summer 2017 -- 2019
\end{itemize}

{\bf Moderating and Discussion Activities:}
\begin{itemize}
  \item Moderator, Organized Oral Session: {\em Advancing Ecological Theory through Synthesis of Long-Term Ecological Research}, ESA, Portland, OR. \hfill 2017
  \item EcoLunch Co-Organizer, Indiana University \hfill August 2015 -- May 2016
  \item Organizer, Metacommunity Reading Group, Indiana University \hfill Summer 2015
\end{itemize}

{\bf Departmental Service:}
\begin{itemize}
  \item Faculty Search Committee, Biology Dept. Mississippi State University \hfill 2023
  \item Zernickow Award Committee, Biology Dept. Mississippi State University \hfill 2022 -- Present
  \item Graduate Recruiting Weekend, Biology Dept. Indiana University \hfill 2014 -- 2020
\end{itemize}

\end{rSection}
\bigskip

\begin{rSection}{Academic Advisors}
Jay Lennon, Indiana University (Ph.D.) \\
Shannon Albeke, University of Wyoming (Postdoc) \\
Lauren Shoemaker, University of Wyoming (Postdoc) 
%Mathew Leibold, University of Texas (B.S.)
\end{rSection}
\bigskip

%-------------------------------------------------------------------------------
%	SOCIETY SECTION
%-------------------------------------------------------------------------------
% \begin{rSection}{Professional Society Membership}
% 
% Ecological Society of America\\
% %Association for the Sciences of Limnology and Oceanography\\
% %International Society for Microbial Ecology\\ 
% Society for Freshwater Sciences\\
% American Society of Naturalists\\ 
% American Society for Microbiology
% 
% \end{rSection}

%-------------------------------------------------------------------------------
%-------------------------------------------------------------------------------
%	REFERENCES SECTION
%-------------------------------------------------------------------------------
% \newpage
% \begin{rSection}{References}
% 
% Jay T. Lennon\\
% lennonj@indiana.edu\\
% Professor, Department of Biology\\
% Indiana University -- Bloomington, IN
% 
% Mathew A. Leibold\\
% mleibold@ufl.edu\\
% Professor, Department of Biology\\
% University of Florida -- Gainesville
% 
% Spencer R. Hall\\
% sprhall@indiana.edu\\
% Professor, Department of Biology\\
% Indiana University -- Bloomington
% 
% \end{rSection}




\end{document}
